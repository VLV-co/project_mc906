
\documentclass[12pt]{article}
\usepackage{amsmath,amsfonts}
\usepackage{geometry}
\usepackage{graphicx}
\usepackage{hyperref}
\geometry{margin=1in}

\title{A Formal Metric for Estimating the Relative Difficulty of 2048 Game Topologies}
\author{Luiz Guilherme Sousa Nascimento}
\date{}

\begin{document}
\maketitle

\section*{1. Introduction}
We aim to define a relative difficulty metric for variants of the 2048 game played on different board topologies. The metric should account for empirical performance, board structure, and stochastic variance, enabling predictive comparisons.

\section*{2. Problem Setting}
Let $G = (V, E)$ represent the game board as a graph with:
\begin{itemize}
  \item $N = |V|$: number of cells
  \item $D$: number of move directions
  \item $\bar{d}$: average node degree (connectivity)
\end{itemize}

Let $T_{\text{max}}$ be the highest tile achieved in a game, and $X = \log_2(T_{\text{max}})$ its log-transformed value. Denote:
\[
\mathbb{E}[X] \quad \text{and} \quad \sigma = \sqrt{\mathrm{Var}(X)}
\]
as the mean and standard deviation across many games on the same topology.

\section*{3. Structural Factors}
More cells ($N$), directions ($D$), and connectivity ($\bar{d}$) yield greater maneuverability. We define:
\[
f(N, D, \bar{d}) = N + \alpha D + \beta \bar{d}
\]
with $\alpha, \beta \geq 0$ expressing the influence of movement and connectivity.

\section*{4. Performance Term with Variance Penalty}
Given the stochastic nature of tile spawns and movement, we define an effective performance measure:
\[
\Phi = \mathbb{E}[X] \cdot (1 + \mu \cdot \sigma)
\]
where $\mu \geq 0$ penalizes high variance as a sign of unstable strategies.

\section*{5. Tile Generation Probabilities}
Let $P_2, P_4$ be the probabilities of spawning tiles 2 and 4. Then:
\[
\mathbb{E}[T_{\text{new}}] = 2P_2 + 4P_4, \quad \mathcal{P} = \frac{1}{\mathbb{E}[T_{\text{new}}]}
\]

\section*{6. Optional Entropy Term}
To capture uncertainty in future board states, we may define:
\[
H = -\sum_i p_i \log p_i, \quad \text{and} \quad (1 + \lambda H)
\]

\section*{7. Final Difficulty Metric}
We define the relative difficulty as:
\[
\boxed{
\mathcal{D}_{\text{topo}} = \frac{f(N, D, \bar{d})}{\Phi} \cdot \mathcal{P} \cdot (1 + \lambda H)
}
\quad \text{with} \quad \Phi = \mathbb{E}[\log_2(T_{\text{max}})] \cdot (1 + \mu \cdot \sigma)
\]

\section*{8. Predicting for New Topologies}
Given a baseline $\mathcal{D}_{\text{ref}}$ and a new topology $(N', D', \bar{d}')$:
\[
\Phi_{\text{new}} \approx \frac{f(N', D', \bar{d}')}{\mathcal{D}_{\text{ref}}} \cdot \mathcal{P}_{\text{new}} \cdot (1 + \lambda H + \mu \sigma)
\]
and:
\[
\mathbb{E}[\log_2(T_{\text{max,new}})] \approx \frac{\Phi_{\text{new}}}{1 + \mu \cdot \sigma}
\]

\section*{9. Conclusion}
This formulation enables principled comparisons between different 2048 topologies. It accounts for spatial capacity, mobility, randomness, and observed empirical performance.

\end{document}
